\documentclass[a4paper]{article}
\usepackage[utf8]{inputenc}
\usepackage[fleqn]{amsmath}
\usepackage{algorithmicx}
\usepackage{algorithm}
\usepackage{algpseudocode}
\usepackage{amssymb}
\usepackage{pdfpages}
\usepackage{mathtools}
\usepackage{amsfonts}
\usepackage{epstopdf}
\usepackage{lastpage}
\usepackage{tikz}
\usepackage{float}
\usepackage{textcomp}
\usetikzlibrary{patterns}
\usepackage{pdfpages}
\usepackage{gauss}
\usepackage{fancyvrb}
\usepackage[table]{colortbl}
\usepackage{fancyhdr}
\usepackage{graphicx}
\usepackage{caption}
\usepackage[margin=1in]{geometry}
\usepackage{subcaption}
\delimitershortfall-1sp
\newcommand\abs[1]{\left|#1\right|}

\def\A{\mathcal{A}}
\def\I{\mathcal{I}}
\def\E{\mathbb{E}}
\definecolor{listinggray}{gray}{0.9}
\usepackage{listings}
\lstset{
	language=,
	literate=
		{æ}{{\ae}}1
		{ø}{{\o}}1
		{å}{{\aa}}1
		{Æ}{{\AE}}1
		{Ø}{{\O}}1
		{Å}{{\AA}}1,
	backgroundcolor=\color{listinggray},
	tabsize=2,
	rulecolor=,
	basicstyle=\scriptsize,
	upquote=true,
	aboveskip={0.2\baselineskip},
	columns=fixed,
	showstringspaces=false,
	extendedchars=true,
	breaklines=true,
	prebreak =\raisebox{0ex}[0ex][0ex]{\ensuremath{\hookleftarrow}},
	frame=single,
	showtabs=false,
	showspaces=false,
	showlines=true,
	showstringspaces=false,
	identifierstyle=\ttfamily,
	keywordstyle=\color[rgb]{0,0,1},
	commentstyle=\color[rgb]{0.133,0.545,0.133},
	stringstyle=\color[rgb]{0.627,0.126,0.941},
  moredelim=**[is][\color{blue}]{@}{@},
}
\newcommand{\comment}[1]{%
  \text{\phantom{(#1)}} \tag{#1}}
\lstdefinestyle{base}{
  emptylines=1,
  breaklines=true,
  basicstyle=\ttfamily\color{black},
}

\pagestyle{fancy}
\def\checkmark{\tikz\fill[scale=0.4](0,.35) -- (.25,0) -- (1,.7) -- (.25,.15) -- cycle;}
\def\E{\mbox{\textbf{E}}}
\def\Pr{\mbox{\textbf{Pr}}}
\newcommand*\circled[1]{\tikz[baseline=(char.base)]{
            \node[shape=circle,draw,inner sep=2pt] (char) {#1};}}
\newcommand*\squared[1]{%
  \tikz[baseline=(R.base)]\node[draw,rectangle,inner sep=0.5pt](R) {#1};\!}
\cfoot{Page \thepage\ of \pageref{LastPage}}
\DeclareGraphicsExtensions{.pdf,.png,.jpg,.eps}
\graphicspath{{image/}}
\author{Ola Rønning (vdl761)\\Tobias Hallundbæk Petersen (xtv657)}
\title{Randomized Algorithms \\ Week 6}
\lhead{RA}
\rhead{Week 6}
\begin{document}
\section{Gametree evaluation}
Binary gametree \(T_{2,k}\): even distance internal nodes are ANDs and odd distanced odds. \(2^{2k}\) leaves, all given binary value. For determinstic algorithms there is an instance st. all leafs must be evaluated.


Randomized algorithm: To evaluate an AND node choose child at random, evaluate subtree of OR-child recursively, if it returns 1 evaluate its sibling otherwise return zero. To evaluate an OR node choose random child, evaluate subtree of AND-child recursively, if it returns 0 evaluate its sibling otherise return 1.

\textbf{Thrm}: We can evaluate any instance of \(T_{2,k}\) with atmost \(3^k\) expected cost. Cost is the number of leafs we must examine.\\
Proof: Proof by induction on k.\\
Base case \(k=1\):\\
Consider AND node with 2 leafs, that evaluate to 0. Atleast one leaf must be 0. With probability \(1/2\) we choose this leaf first, otherwise we must check both. This gives the expectation \(1/2 * 1 + 1/2 * 2 = 3/2< 3\).
Consider OR node with 2 leafs, that eval to 1. Atleast one leaf must be 1. With probability \(1/2\) we choose this first, otherwise we must check both. This gives expectation \(1/2 * 1 + 1/2 * 2 = 3/2 < 3\). If we have one and 0 for AND and OR respectively we must examine both which gives cost \(2<3\).\\
We now assume that expected cost of evaluating any instance of \(T_{2,k-1}\) is \(3^{k-1}\). Consider first root is OR that evaluates to 1. Then atleast one child must evaluate to 1, with prob \(1/2\) this is choosen first with expected cost \(3^{k-1}\) otherwise we check both children with expected cost \(2*3^{k-1}\). Gives expectation 
\[1/2*3^{k-1} + 1/2 * 2 * 3^{k-1} = 3/2 * k^{k-1}\]
If OR root evaluate to 0, we have expected cost \(2*3^{k-1}\).\\
Consider root is AND node. Then if it evaluates to 1 both OR-children must be evaluated, by previous analysis expectation is \(2*3/2*k^{k-1} = 3^k\). If it evaluates to 0 we have probabilty \(1/2\) that 0 node choosen first with expectation otherise we must pay the price of evluating an OR node that evaluates to 1, so
\[2*3^{k-1} + 1/2*3/2*3^{k-1} \leq 3^k\] 
Since \(n=4^k\) our expected runtime is atmost \(n^{\log_43}\ \leq n^{0.793}\).
\section{The minmax principle}
\textbf{Thrm}:
For all distributions $p$ over $\I$ and $q$ over $\A$, where C(I,A) is the cost of running algorithm A on input I.
$$
\min_{A \in \A} \E[C(I_p, A)] \leq \max_{I\in\I} \E[C(I,A_q)]
$$
ie. the expected running time of the optimal deterministic algorithmen for an arbitrarily chosin input distribution $p$ ias a lower bound on the expected running time of the optimal Las Vegas randomized algorithm for the problem.\\
Proof:
We define $q \in Q$ as all the input probabilities, and $p\in P$ as all the algorithm probabilities, $A_q$ is the algorithm corresponding to probability $q$ and $I_p$ as the input corresponding to probability $p$.\\
We let $c = \max_{I\in\I} \E[C(A,I)]$. For every input $I$, we have $\sum_{q \in Q} q C(A_q,I) \leq c$. Therefore we have
$$\sum_{p \in P} p \sum_{q \in Q} q C(A_q,I_p) \leq c.$$
As all terms are non-negative we can switch order of summation giving us
$$\sum_{q \in Q} q \sum_{p \in P} p C(A_q,I_p) \leq c.$$
By the Pigeonhole principle there must exist an algorithm $A_q$ such that $\sum_{p \in P} p C(A_q,I_p) \leq c$, concluding the proof.
\section{Yao on Game Tree}
A tree $T_{2,k}$ is equivalent to a balanced binary tree of whose leaves are at distance 2k from the root, where nodes compute the NOR function.
We want to analyse this tree of NORs with depth 2k using Yao's Minimax principle to prove a lower boundon the expected number of leaves evaluated by any randomized algorithm.
We here chose the input probability, i.e. the probability that a leaf is 1, to be $p = (3 - \sqrt{5})/2$. We note that if each input to a NOR node is independently 1 iwht probability p, then the probability that its output is 1 is the probability that both its inputs are 0, which is:
$$
(1 - p)^2 = \left(\frac{\sqrt 5 - 1}{2} \right)^2 = \frac{3-\sqrt{5}}{2} = p
$$
\paragraph{Prop 2.7 (USED WITHOUT PROOF)} Let $T$ be a NOR tree each of whose leaves is independently set to 1 with probability q for a fixed value $q \in [0,1]$. Let $W(T)$ denote the minimum over all deterministic algorithms, of the expected number of steps to evaluate $T$. Then there is a depth-first pruning algorithm whose expected number of steps to evaluate $T$ is $W(T)$\\
For a depth first pruning algorithm evaluating this tree, we let $W(h)$ be the expected number of leaves it inspects in determining the value of a node at distance $h$ from the leaves, then
$$
W(h) = W(h-1) + (1-p) * W(h-1)
$$
If we let $h$ be $\log_2(n)$ and solve we get $W(h) \geq n^{0.694}$
\section{Randomness \& non-uniforimity}
Boolean circuit with n input is a digraph where i) there are n input nodes with in-deg 0 and one output node with out-deg 0. ii) All nodes not input or output have one of the boolean functions AND,OR,NOT b(v) associated with it. iii) circuit compute function on inputs in intuitive manner. iv) size of circuit is \(|V|\).\\
A random boolean circuit has additional inputs \(r_1,..,r_m\) that take on values in \(\{0,1\}\). Computes f on inputs if i) for inputs where f(x1,x2,..,xn)=0, output is 0. ii) for input where f(x1,x2,..,xn)=1, output is 1 with atleast probabilty 1/2.\\
Circuit family of f: Is a sequence of circuits such that the i'th circuit computs \(f_i\) on i inputs. has poly size if \(|V|\) is bounded by polynomial of n. Randomized circuit fam: same with requires i and ii.\\ 
\textbf{Thrm}: If a Boolean function has a randomized polynomial-sized family, then it has a polynomial-sized circuit family.\\
Proof: For a matrix M with \(2^n\) rows, one for each possible input from \(\{0,1\}^n\). and \(2^m\) columns, one for each possigl3 m-tuple that \(r_i\) can assume. \(M_{jk}\) is 1 if \(r_1,..,r_m\) corresponding to column k is a witness for input \(x_1,..x_n\) otherwise set it to zero. Remove all rows from M where \(f_n=0\). By definition atleast half the entries of every survining row of M equal 1. Therefore, there must be a column with atleast half its entries 1; int otherwords there is an assigmnet of 0s and 1s to \(r_i\) that witness atleast half of all outputs. Build circuit \(T_1\) which is copy of \(C_n\) with hardwired randon inputs to the witness. Remove all rows witnessed by previous random assignment. The remaining rows was not witness and so must have half their entris 1s. repeat the construction, with \(T_2,...,T_n\). since we reduce by a factor two each time we can atmost due this procedure \(n+1\) times, the or reduction of outputs is a deterministic algorithm. 
\end{document}
