\documentclass[a4paper]{article}
\usepackage[utf8]{inputenc}
\usepackage[fleqn]{amsmath}
\usepackage{algorithmicx}
\usepackage{algorithm}
\usepackage{algpseudocode}
\usepackage{amssymb}
\usepackage{pdfpages}
\usepackage{mathtools}
\usepackage{amsfonts}
\usepackage{epstopdf}
\usepackage{lastpage}
\usepackage{tikz}
\usepackage{float}
\usepackage{textcomp}
\usetikzlibrary{patterns}
\usepackage{pdfpages}
\usepackage{gauss}
\usepackage{fancyvrb}
\usepackage[table]{colortbl}
\usepackage{fancyhdr}
\usepackage{graphicx}
\usepackage{caption}
\usepackage[margin=1in]{geometry}
\usepackage{subcaption}
\delimitershortfall-1sp
\newcommand\abs[1]{\left|#1\right|}

\definecolor{listinggray}{gray}{0.9}
\usepackage{listings}
\lstset{
	language=,
	literate=
		{æ}{{\ae}}1
		{ø}{{\o}}1
		{å}{{\aa}}1
		{Æ}{{\AE}}1
		{Ø}{{\O}}1
		{Å}{{\AA}}1,
	backgroundcolor=\color{listinggray},
	tabsize=2,
	rulecolor=,
	basicstyle=\scriptsize,
	upquote=true,
	aboveskip={0.2\baselineskip},
	columns=fixed,
	showstringspaces=false,
	extendedchars=true,
	breaklines=true,
	prebreak =\raisebox{0ex}[0ex][0ex]{\ensuremath{\hookleftarrow}},
	frame=single,
	showtabs=false,
	showspaces=false,
	showlines=true,
	showstringspaces=false,
	identifierstyle=\ttfamily,
	keywordstyle=\color[rgb]{0,0,1},
	commentstyle=\color[rgb]{0.133,0.545,0.133},
	stringstyle=\color[rgb]{0.627,0.126,0.941},
  moredelim=**[is][\color{blue}]{@}{@},
}
\newcommand{\comment}[1]{%
  \text{\phantom{(#1)}} \tag{#1}}
\lstdefinestyle{base}{
  emptylines=1,
  breaklines=true,
  basicstyle=\ttfamily\color{black},
}

\pagestyle{fancy}
\def\checkmark{\tikz\fill[scale=0.4](0,.35) -- (.25,0) -- (1,.7) -- (.25,.15) -- cycle;}
\def\E{\mbox{\textbf{E}}}
\def\Pr{\mbox{\textbf{Pr}}}
\newcommand*\circled[1]{\tikz[baseline=(char.base)]{
            \node[shape=circle,draw,inner sep=2pt] (char) {#1};}}
\newcommand*\squared[1]{%
  \tikz[baseline=(R.base)]\node[draw,rectangle,inner sep=0.5pt](R) {#1};\!}
\cfoot{Page \thepage\ of \pageref{LastPage}}
\DeclareGraphicsExtensions{.pdf,.png,.jpg,.eps}
\graphicspath{{image/}}
\lhead{RA}
\rhead{Chap 4}
\begin{document}
\section{Chernoff Bound}
Bounds the probability of deviating far from the mean. First we bound devaition above, then below. We consider moment generating function \(E[e^{tX}]\) rather than \(E[X]\). We work with poisson trials, indpendent \(p_i\) prob that\(X_i=1\).\\
Thrm Chernoff above: Let \(X_{i\leq n}\) be indp poisson trials, st. \(\Pr[X_1 = 1] = p_i\). Then for \(\delta > 0\) 
\[\Pr[X>(1+\delta)\mu] < \left[\frac{e^\delta}{(1+\delta)^{1+\delta}}\right]\]

\section{Parallel routing}
Problem: consider network of N nodes. \(v_i\) is the package from processor \(i\), \(d(i)\) is destination. Only consider case when \(d(i)\) are all distinct, ie. permutation of \([N]\). How many steps required to route arbitrary permutation of \(d(1),..,d(N)\).
Consider case where all edges leaving vertex have queue, a route is a sequence of edges to take from source to destination. Oblivous algorithm, all routes are determined only by destination. 
Thrm: For N nodes with out-degree d, there is deterministic oblivous routing algorithm there is a instance of permutation routing alg. requiring \(\Omega(\sqrt{N/d})\) steps.

Boolean hypercube: \(N=2^{dim}\) nodes. node \(i \in \{0,1\}^n\) adjacent to nodes with index obtained by single boolean flip. fx 010 adj. to \(011,000,110\). If j'th bit flipped out egde labeled j.\\
2 phase ran alg:
\begin{enumerate}
\item pick random intermediate destination \(\sigma(i)\) from [N]. \(v_i\) travels to \(\sigma(i)\).
\item When all packages have arrived at intermediate destination, all packages travel to destination \(d(i)\).
\end{enumerate}
Route determined by bit-fixing strategy: send \(v_i\) at \(v_j\) along the edge most sig. bit of \(v_j - d(i)\). Note \(\sigma\) is not permutation, as all destination chosen indpendently. Use FIFO for queueing.\\
Phase 1 analysis, \(\rho_i\) is phase one route for \(v_i\). \(|\rho_i| \leq n+delay\).\\
\textbf{Lemma}: delay of \(v_i\) with route \(e_1,..,e_k\) is atmost \(|S|\). Where S is the set of packages which route pass through atleast one of \(e_1,..,e_k\).\\
Proof: A package in S leaves \(\rho_i\) at time step step at which it traverses an edge in \(\rho_i\) for the last time. Lag of a package traversing \(e_j\) at time t is \(t-j\). Init lag of \(v_i=0\), lag of \(v_i\) as traverses \(e_k\) is delay. If lag of \(v_i\) reaches l+1 a package in S leaves \(\rho_i\) with l. We see this since there must be a package from S that traveses \(e_j\), otherwise \(v_i\) could just traverse \(e_j\). Our delay policy FIFO means even if that package continue to travel along \(\rho_i\) it will not delay \(v_i\) again.  Let t' be the last time at which a package in S leaves with lag l. So there exist package v traversing \(e_{j'}\) at time t' st \(t'-e_{j'} = l\). Since v follows \(e_{j'}\) there is some w in S that follows \(e_{j'}\) at time t'. w must leave \(\rho_i\) at t': if not, some package follows \(e_{j'+1}\) at time \(t'+1\) with lag l, and so t' is not maximal. We charge increase from l to l+1 to w, which will not delay \(v_i\) again since it leaves \(\rho_i\). Thus each member of \(S\) can each be charge atmost one delay, establishing the lemma.\\
Let \(H_{ij}=1\) iff \(\rho_i\) and \(\rho_j\) share atleast one edge. Delay \(v_i\) is at most \(\sum_1^N H_{ij}\). Since routes are chosen randomly \(H_{ij}\) are poisson trial for \(i\neq j\). Hence to bound delay using Chernoff, we can bound sum over \(H_{ij}\).\\
Consider edge \(e\) in the hypercube, let \(T(e)\) be the number of routes that traverse \(e\). Fix \(\rho_i\) with \(|\rho_i| = k \leq n\), then
\[\sum_1^N H_{ij} \leq \sum_1^k T(e_l) \Rightarrow E[\sum_1^N H_{ij}] \leq \sum_1^k E[T(e_l)]\]
The expected length of \(\rho_j\) is \(n/2\). So the the expected total length is \(Nn/2\). There are \(Nn\) nodes \(\Rightarrow E[T(e)]= 1/2\) for all edges \(e\).
\[E[\sum_1^N H_{ij}] \leq k/2 \leq n/2\]
By Chernoff bound the probability that \(\sum_1^N H_{ij}\) exceeds 6n \(<\) \(2^{-6n}\). Summing over all packets we get probability \(2^{-5n}\) that none exceed 6n delay. Adding the route length we get with probability \(1-2^{-5n}\) every package reaches phase 1 dest. atmost 7n steps.\\
For phase to we run same anlysis, only source is chosen at random. Hence we have that with for package to reach dest in either phase is \(2*2^{-5n} < 1/N\) for \(n>1\).\\
With probaility at least \(1-(1/N)\) every packet reaches its destination in atmost 14n steps. 
\section{Wiring problem}
Problem: global wiring in gate-arryas.\\
Gate-array: \(\sqrt{n}\times\sqrt{n}\) matrix with elements number \([n]+1\).\\\
A net: is a set of gates that must be connect.\\
Problem to solve: Given a set of nets. Consider only nets of size 2. We want to specify a physical path for each net.\\
The boundary between two gates can accomidate atmost \(w\) wires crossing it. We want to solve global wiring st. no boundary has more \(w\) wires crossing it.\\
We consider only one bend routes\\
We cast problem to 0-1 linear program. For net i we have indicators \(x_{i0}=1\) iff net i goes horizontal first and \(x_{i1}=1\) iff we go verical first, starting from the leftmost gate.\\
For each boundary in the gatearray
\[T_{b0} = \{i | \emph{net} i \emph{passes through } b \emph{ if } x_{i0} = 1\}\]
and 
\[T_{b1} = \{i | \emph{net} i \emph{passes through } b \emph{ if } x_{i1} = 1\}\]
Linear program:
\[\min w \emph{ where }x_{i0},x_{i1} \in \{0,1\}\]
subject to
\begin{align}
x_{i0} + x_{i1} &= 1\\
\sum_{i \in T_{b0} x_{i0}} + \sum_{i \in T_{i1}} x_{i1} &\leq w
\end{align}
Problem is NP-hard. Do relaxation to \(x_{i0},x_{i1} \in [0,1]\) and use randomized rouding to 1 with probabilties equal to relaxed solution.\\
\textbf{Thrm} Let \(\epsilon\) be a real number st. \(0<\epsilon<1\). Then with probability \(1-\epsilon\) the global wiring \(S\) produced by randomized rounding satisfies
\[w_S \leq \hat{w}(1+\Delta^+(\hat{w},\epsilon/2n)) \leq w_0(1+\Delta^+(w_0,\epsilon/2n))\]
Note \(w_0\) is optimal solutions (max \(w\) possible).\\
Proof: Consider a boundary \(b\); since the solutoin of the linear program satisfy its constrains, we have
\[\sum_{i \in T_{b0}} \bar{x}_{i0} + \sum_{i\in T_{b1}} \bar{x}_{i1}\]
The number of wires passing thorugh \(b\) in the solution \(S\)  is
\[w_s(b) = \sum_{i \in T_{b0}} \bar{x}_{i0} + \sum_{i\in T_{b1}} \bar{x}_{i1}\]
 But \(\bar{x}_{i0}\) and \(\bar{x}_{i1}\) are Poisson trials with probabilty \(\hat{x}_{i0}\) and \(\hat{x}_{i1}\), respecively. Further, \(\bar{x}_{i0}\) and \(\bar{x}_{i1}\) are each indpendent of \(j\) for \(i\neq j\). Therefore \(w_S(b)\) is the sim of independent Poisson trials and by previous two equations
 \[E[w_S(b)] = \sum_{i \in T_{i0}} E[\bar{x}_{i0}] + \sum_{i\in T_{b1}} E[\bar{x}_{i1}] = \sum_{i \in T_{i0}} \hat{x}_{i0} + \sum_{i\in T_{b1}} \hat{x}_{i1} \leq \hat{w}\]
By definition of \(\Delta^+(\mu,\epsilon)\)
\[\Pr[w_s(b) > \hat{w}(1+\Delta^+(\hat{w},\epsilon/2m)] \epsilon/2n\]

\end{document}
